\documentclass[10pt,a4paper,reqno]{amsart}

\usepackage[T1]{fontenc}
\usepackage{fullpage}
\usepackage{macros}

\begin{document}
\bibliographystyle{alpha}

\noindent \textit{Algebraic Number Theory, Math 421}

\noindent \textit{Instructor: Sreekar M. Shastry}

\noindent \emph{Notes on the notion of support}

\bigskip

\begin{ap}
Let $f\in \Z$. We want to regard $f$ as a function on \Spec{\Z} which assigns
to a prime $p$, something\ldots
\end{ap}

\begin{defn}
Let $R$ be a ring and $M$ be an $R$-module. We define the sheaf associated to
$M$ on $\Spec{R}$, denoted by $\smash{\widetilde{M}}$ as follows. For each
$\mathfrak{p} \subset R$, let $M_\mathfrak{p}$ be the localization of $M$ at
$\mathfrak{p}$ (i.e.~localization of $R$ w.r.t. the multiplicative set
$M-\mathfrak{p}$). For any open set $U\subset \Spec{R}$ we define the group
$\smash{\widetilde{M}}(U)$ to be { \[\smash{\widetilde{M}}(U) := \{s : U
\rightarrow \coprod_{\mathfrak{p}\in U} M_\mathfrak{p} : \forall \mathfrak{p}
\in U \ \exists V \ni \mathfrak{p}, m\in M, f\in R\ \forall \mathfrak{q} \in V,
f\notin \mathfrak{q}, s(\mathfrak{q}) = m/f \in M_\mathfrak{q}\}.\]} We make
$\smash{\widetilde{M}}$ into a sheaf by means of the obvious restriction maps.

In words, $\smash{\widetilde{M}}(U)$ is the set of functions $s : U \rightarrow
\coprod_{\mathfrak{p}\in U} M_\mathfrak{p} $ such that for each
$\mathfrak{p}\in U$, $s(\mathfrak{p})\in M_\mathfrak{p}$ and such that $s$ is
locally given by a fraction $m/f$ with $m\in M$ and $f\in R$ nonvanishing in a
neighborhood of $\mathfrak{p}$.
\end{defn}

\begin{ap}
Thus in our setting, $f\in \Z = H^0(\Spec{\Z}, \mathscr{O})$, the \Z-module $M$
in question is taken to be the \emph{quotient} $\Z/(f)$; note that we may
define $f(p) := $ the image of $f$ in $M_{(p)}$ or even the image of $f$ in
$\mathbb{F}_p$ but it turns out that this is not the notion we are looking for.

We are looking for a notion of support of the ideal $M$:

\[\text{supp}(M) := \{\mathfrak{p} \in \Spec{R} : M_\mathfrak{p} \neq 0\}.\]
This is to remind one of the notion of the support of a function $f : X
\rightarrow \R$, \[\text{supp}(f) := \{x \in X : f(x) \neq 0 \}^{-}.\]

Now, $M_{(p)} = 0$ iff there is an $s\notin (p)$ such that $sM = 0$. There is
such an $s$ iff $f|s$. Thus there is such an $s$ iff $p\ndiv s$ and $f|s$. Thus
\[\{p \in \Spec{\Z} : M_{(p)} = 0\} = \{p \in \Spec{\Z} : p \ndiv f\}.\] This
set is the complement of $\text{supp}(M)$ and therefore \[\text{supp}(M) = \{p
: p|f\}.\]

(If we used the definition of $f(p) := $ the image of $f$ in $\mathbb{F}_p$
then the support of $f$ would be the set of primes which do \emph{not} divide
$f$.)
\end{ap}

\begin{ap}
Let us contrast with the case of $R = \C[z]$. Let $f\in \C[z]$ and $M :=
\C[z]/(f)$. Then by the same reasoning as above we have \[\text{supp}(M) =
\{(z-w) \in \Spec{\C[z]} : (z-w) | f\} = \{w \in \C : f(w) = 0\} =
f^{-1}(0)\] (some of the equalities only indicate bijections, of course).

Thus the support of the module $M := \C[z]/(f)$ is precisely the
\emph{complement} [sic] of the support of the function $f : \C \rightarrow \C$.
\end{ap}

\begin{ap}
Given $f\in \C[z]$ we of course may regard $f$ as a entire holomorphic function
$f: \C\rightarrow \C$. Thinking in terms of scheme theory, we may interpret
this function as follows. Let $X := \Spec{\C[z]}$ be the affine line over \C{}.
Then a closed point of $X$ corresponds to a maximal ideal $\mathfrak{m}_w :=
(z-w)$ for some $w\in \C$ because \C{} is algebraically closed. The residue
field of $X$ at $\mathfrak{m}_w$ is a copy of $\C$ which we shall denote
$\C_w$. Thus we may consider the set theoretic function, \[ f_\# : X
\rightarrow \coprod_{w\in \C} \C_w \] which sends $\mathfrak{m}_w$ to $f
\imod{\mathfrak{m}_w}$. This function is defined only on closed points. This
function is nonzero precisely on the set of maximal ideals which correspond to
those $w$ such that $f(w) \neq 0$. Thus the naive notion of support of the
function $f_\#$ coincides with the notion of support from analysis.
\end{ap}

\begin{ap}
Let $n \in \Z = H^0(\Spec{\Z}, \mathscr{O}) $. Proceeding as above we may
contemplate \[ n_\# : \Spec{\Z} \rightarrow \coprod_{p} \mathbb{F}_p \] which
sends the ideal $(p)$ to the residue class of $n\imod{p}$ in $\mathbb{F}_p$.
Then the set theoretic support of $n_\#$ is precisely the set of primes which
do \emph{not} divide $n$.
\end{ap}

\begin{ap}
Thus we see that it is possible to generalize the notion of support of a
function $f$ as found in analysis, but this is precisely the complement of the
notion of support as is used in commutative algebra and algebraic geometry,
where one is interested in the support of the sheaf of modules $R/(f)$ where
$f$ is in the ring of functions $R$ on a geometric object.

Try not to get confused\ldots
\end{ap}

\bigskip

\noindent References: Hartshorne, Atiyah-MacDonald, Eisenbud.

\end{document}

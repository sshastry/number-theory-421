\documentclass[10pt,a4paper,reqno]{amsart}
\usepackage[T1]{fontenc}
\usepackage{macros}

\begin{document}
\bibliographystyle{alpha}

\noindent \textit{Algebraic Number Theory, Math 421}

\noindent \textit{Instructor: Sreekar M. Shastry}

\noindent \textit{Notes on the ideal class group and the unit group}

\section{Finiteness of the Class Group}

We use the usual notation, $K, L/K, G, \OK, \OL, P, Q, \dots$

\begin{thm}
Given $K$, there is a positive real number $\lambda$, depending only on $K$
such that every nonzero ideal $I\subset \OK$ contains a nonzero element
$\alpha$ with \[|\Nm_{K/\Q}(\alpha)| \le \lambda \norm{I}.\]
\end{thm}
\begin{proof}
Let \(\{\alpha_i\}_{i=1}^n\) be an integral basis for \OK{} and
$\{\sigma_j\}_{j=1}^n$ be the set of embeddings $K\hookrightarrow \C$. We will
show that \[\lambda := \prod_{i=1}^n \sum_{j=1}^n |\sigma_i(\alpha_j)|\] will
do the job.

For an ideal $I$, let $m$ be the unique positive integer such that \[m^n \le
\norm{I} < (m+1)^n\] and consider the $(m+1)^n$ elements of \OK:
\[\left\{\sum_{j=1}^n m_j\alpha_j : m_j \in \Z, 0 \le m_j \le m \right\}\] (we
are secretly using the fact that $\mathrm{char}(K)=0$). Two elements of this
set must be congruent mod $I$ since the set has cardinality greater than
\norm{I}; we take their difference and obtain a nonzero element of $I$ of the
form \[\alpha = \sum_{j=1}^n m_j\alpha_j \text{ s.t. } m_j\in\Z, |m_j| \le m.\]
The following chain of inequalities completes the proof: \[|\Nm_{K/\Q}(\alpha)|
= \prod_{i=1}^n |\sigma_i(\alpha)| \le \prod_{i=1}^n \sum_{j=1}^n
m_j|\sigma_i(\alpha_j)| \le m^n\lambda \le \norm{I}\lambda.\]
\end{proof}

\begin{cor}
Every ideal class of \OK{} contains an ideal $J$ with $\norm{J} \le \lambda$
(the same $\lambda$ as above).
\end{cor}
\begin{proof}
Given an ideal class $C$, fix a representative $I$ of $C^{-1}$ and choose
$\alpha\in I$ as in the theorem. We have $I\supset (\alpha)$ so that $(\alpha)
= IJ$ for some $J$ (by \cite[Theorem 15,p.57]{M} and its proof). Then $J$ is in
the class of $C$. Now we use the theorem to the effect that \[\norm{(\alpha)} =
|\Nm_{K/\Q}(\alpha)|\] to see that \[| \Nm_{K/\Q}(\alpha) | = \norm{(\alpha)} =
\norm{I}\norm{J} \le \lambda\norm{I}. \]
\end{proof}

\begin{cor} The class group, $\Cl{\OK}$, is finite.
\end{cor}
\begin{proof}
If $J = \prod P_i^{n_i}$ then $\norm{J} = \prod \norm{P_i}^{n_i}.$ Thus only
finitely many \emph{ideals} can satisfy $\norm{J} \le \lambda$ since this
inequality implies the same inequality for every $P | J$ and thus places bounds
on the powers to which they can occur.
\end{proof}

\begin{eg}
Let us use the above to show that $\Z[\sqrt{2}]$ is a PID or in other words
that $\Cl{\Z[\sqrt{2}]}= \{1\}$.

The integral basis is $\{1,\sqrt{2}\}$ (since $2 \equiv 2 \imod{4}$!) and the
above proof produces $\lambda = (1+\sqrt{2})^2 = 1 + 2\sqrt{2} + 2$. This
number is between 5 and 6 and therefore every ideal class contains an ideal $J$
with $\norm{J} \le 6.$ Thus the support of $J$ must be contained in the set of
primes lying over $2,3,5$. We factor each of $2.\OK, 3.\OK, 5.\OK:$ $2.\OK =
\left( \sqrt{2} \right)^2$ while $3.\OK$ and $5.\OK$ are inert because $2$ is a
quadratic nonresidue mod 3 and 5 (see \cite[Theorem 25,p.74]{M}). Therefore the
only ideals $J$ with $\norm{J} \le 5$ are $\OK, \left( \sqrt{2} \right),$ and
$2.\OK.$ It follows that every ideal in \OK{} is principle.
\end{eg}

\section{Minkowski's Theorem}

\begin{ap}
We now seek to improve the constant $\lambda$. We will accomplish this by
embedding $\OK$ into $\R^n$ and applying general geometric results.

Let $\{\sigma_1,\dots,\sigma_r\}$ be the embeddings $K\hookrightarrow \R$ and
$\{\tau_1, \overline{\tau}_1,\dots,\tau_s,\overline{\tau}_s \}$ be the
embeddings $K \hookrightarrow \C$ which do not factor through $\R \subset \C$.
Thus $r+2s = n = [L:K].$ We obtain a map $\iota: K\rightarrow \R^n$ by \[\alpha
\mapsto (\sigma_1(\alpha), \dots, \sigma_r(\alpha), \Re{\tau_1(\alpha)},
\Im{\tau_1,(\alpha)}\dots, \Re{\tau_s(\alpha)}, \Im{\tau_s(\alpha)}).\] Observe
that it is a homomorphism of abelian groups with trivial kernel.
\end{ap}

\begin{thm}
The mapping $\iota : K \rightarrow \R^n$ sends \OK{} isomorphically (as abelian
groups) onto a lattice in $\R^n$. A fundamental domain for this lattice has
volume \[\frac{1}{2^s}\sqrt{|\disc{\OK}|}.\]
\end{thm}
\begin{proof}
Let us see that $\iota(\OK)$ is a lattice $\Lambda$ in $\R^n$ (a lattice is by
definition the \Z-span of an \R-basis for $\R^n$). To see this let
$\{\alpha_i\}_{i=1}^n$ be an integral basis for \OK. These generate \OK{} over
\Z{} and therefore their images in $\R^n$ generate $\Lambda$ over \Z. We must
show that this generating set is linearly independent over \R{}. Consider the
$n\times n$ matrix the $i$th row of which is $\iota(\alpha_i)$ (recall that $n
= r+2s$):
{\footnotesize
\[ A := \begin{pmatrix}
\sigma_1(\alpha_1) & \cdots & \sigma_r(\alpha_1) & \Re{\tau_1(\alpha_1)} &
\Im{\tau_1(\alpha_1)} & \cdots & \Re{\tau_s(\alpha_1)} & \Im{\tau_s(\alpha_1)}\\
\vdots & & \vdots & \vdots & \vdots & & \vdots & \vdots \\
\sigma_1(\alpha_n) & \cdots & \sigma_r(\alpha_n) & \Re{\tau_1(\alpha_n)} &
\Im{\tau_1(\alpha_n)}& \cdots & \Re{\tau_s(\alpha_n)} & \Im{\tau_s(\alpha_n)}
\end{pmatrix}. \]} Elementary column operations transform this matrix
into
\[ B := \begin{pmatrix}
\sigma_1(\alpha_1) & \cdots & \sigma_r(\alpha_1) & {\overline{\tau}_1(\alpha_1)} &
{\tau_1(\alpha_1)} & \cdots & {\overline{\tau}_s(\alpha_1)} & {\tau_s(\alpha_1)}\\
\vdots & & \vdots & \vdots & \vdots & & \vdots & \vdots \\
\sigma_1(\alpha_n) & \cdots & \sigma_r(\alpha_n) & {\overline{\tau}_1(\alpha_n)} &
{\tau_1(\alpha_n)}& \cdots & {\overline{\tau}_s(\alpha_n)} & {\tau_s(\alpha_n)}
\end{pmatrix} \] so that we have
\[\det(A) = \frac{1}{(2i)^s} \det(B).\]
Now, $\det(B)$ is the discriminant of the number field $K$ (in particular it is
nonzero by \cite[Theorem 7,p.25]{M}). On the other hand, the volume of the
fundamental domain in question is $|\det(A)|$.
\end{proof}

\begin{ap}
Let us define a fundamental domain for a lattice in Euclidean space with
\Z-basis $\{v_1,\dots,v_n\}$ to be \[\left\{ \sum_{i=1}^n a_i v_i : a_i \in
[0,1) \subset \R \right\}.\] From linear algebra, we know that the volume of
the fundamental domain is the determinant of the matrix with rows $\{v_i\}$.
This volume is moreover the volume (with respect to the Riemannian metric
induced by pushforward from $\R^n$) of the compact torus $\R^n/\Lambda$.

For a sublattice $\Lambda' \subset \Lambda$, the group $\Lambda/\Lambda'$ is
finite and we have \[\vol(\R^n/\Lambda') =
\vol(\R^n/\Lambda)|\Lambda/\Lambda'|.\] We apply this to an ideal $I \subset
\OK$ to obtain \[\vol(\R^n/\Lambda_I) =
\vol(\R^n/\Lambda_{\OK})|\Lambda_{\OK}/\Lambda_I| =
\frac{1}{2^s}(|\disc{\OK}|)^{1/2}\norm{I}.\]

We now define a ``norm'' on $\R^n$ for $x = (x_1,\dots,x_n)$ by \[\Nm(x) := x_1
\cdots x_r \underbrace{(x_{r+1}^2+x_{r+2}^2)\cdots (x_{n-1}^2+x_n^2)}_{s\text{
factors}}\] where $n = r + 2s$. Thus we have \[\Nm(\iota(x)) = \Nm_{K/\Q}(x)\]
\end{ap}

\begin{thm}\label{Minkowski}
Every lattice $\Lambda \subset \R^n$ contains a nonzero point $x$ such that
\[|\Nm(x)| \le \frac{n!}{n^n}\left( \frac{8}{\pi} \right)^s
\vol(\R^n/\Lambda).\]
\end{thm}

\begin{cor}
A nonzero ideal $I\subset \OK$ contains a nonzero element $\alpha$ such that
\[|\Nm_{K/\Q}(\alpha)| \le \frac{n!}{n^n} \left( \frac{4}{\pi} \right)^s
\norm{I} \sqrt{|\disc{\OK}|}.\]
\end{cor}
\begin{proof}
Apply the above theorem with $\Lambda = \Lambda_I$, use the earlier result that
\[\vol(\R^n/\Lambda_{\OK}) = 2^{-s}|\disc{\OK}|^{1/2},\] and the fact that
$\vol(\R^n/\Lambda_{I}) = \norm{I}\vol(\R^n/\Lambda_{\OK})$.
\end{proof}

\begin{cor}
Each ideal class of \OK{} has a representative $J$ with \[\norm{J} \le
\frac{n!}{n^n} \left( \frac{4}{\pi} \right)^s \sqrt{|\disc{\OK}|}. \]
\end{cor}
\begin{proof}
Given an ideal class $C$ choose an ideal $I\in C^{-1}$ and find an $\alpha$ as
in the previous corollary. Now, there is a $J$ in $C$ such that $(\alpha) = IJ$
so that the previous corollary gives \[|\Nm_{K/\Q}(\alpha)| = \norm{I} \norm{J}
\le \frac{n!}{n^n} \left( \frac{4}{\pi} \right)^s \norm{I}
\sqrt{|\disc{\OK}|},\] as required.
\end{proof}

\begin{ap}
The factor $\frac{n!}{n^n} \left( \frac{4}{\pi} \right)^s$ is called
Minkowski's constant and it decays quickly with respect to $n$. As an example
of its use, let us show that $\Q[e^{2\pi i/5}]$ is has trivial class group.
Every ideal class contains a $J$ with $\norm{J} \le \frac{15\sqrt{5}}{2\pi^2} <
2.$ Thus every ideal class contains \OK{} itself, or in other words, every
ideal class divides the trivial ideal class.
\end{ap}

We now begin the the proof of the theorem.

\begin{lem}[Blichtfeldt's Lemma]
Let $\Lambda$ be an $n$-dimensional lattice in $\R^n$ and let $E$ be a convex,
measurable, centrally symmetric subset of $\R^n$ such that \[\vol(E) > 2^n
\vol(\R^n/\Lambda).\] Then $E$ contains a nonzero point of $\Lambda$. If $E$ is
also compact, then the strict inequality can be weakened to $\ge$.

(Recall that a subset of $\R^n$ is convex if for any $x,y$ in it and $\alpha\in
[0,1]$, $\alpha x + (1-\alpha)y$ is also in it. Centrally symmetric means $-E
= E$.)
\end{lem}
\begin{proof}
Let $F$ be a fundamental domain for $\Lambda$. Then $\R^n$ is the disjoint
union of the translates $x+F, x\in \Lambda$. Thus we have \[\half E =
\coprod_{x\in \Lambda} \half E \cap (x+F).\] Under the hypothesis of strict
inequality, we have
\begin{align*}
    \vol(F) & < 2^{-n}\vol(E) \\
    &= \vol\left( \half E \right) \\
    &= \sum_{x\in \Lambda} \vol\left( {\half E \cap (x+F)} \right)\\
    &= \sum_{x\in \Lambda} \vol\left( {\left( \half E - x \right)
    \cap F} \right).
\end{align*} On the other hand, if the sets $ (\half E - x )\cap F $ were all
pairwise disjoint then the last equation sum would be at most $\vol(F)$. Thus
there exist $x,y\in\Lambda$ such that $\half E -x$ and $\half E -y$ intersect.
Then $x-y \in \Lambda\smallsetminus \{0\}.$

Now, convexity and central symmetry imply that $0\in E$, $\half E \subset E$,
and $E = \half E + \half E$ (the set of all possible sums). Let $z = e'-x = e''
-y $ where $e',e'' \in \half E$. Then $x-y = -z + e' - (-z + e'') = e'+e''\in
E$. This completes the proof under the strict inequality hypothesis.

Next, suppose that $E$ is compact hence closed and bounded and now weaken the
hypothesis to $\vol(E) \ge 2^n \vol(\R^n/\Lambda).$ For each $m=1,2,\dots$ the
first part of the theorem ensures that $(1+1/m)E$ contains some nonzero point
$x_m\in \Lambda$. The $x_m$ are all in $2E$ and since the sequence $\{x_m\}$ is
discrete, it consists of only finitely many distinct points. Thus one of them,
say $x_{m_0}$, is in infinitely many of the $(1+1/m)E$ and thus is in
$\overline{E} = E.$
\end{proof}

\begin{cor}[of the lemma]
Suppose there is a compact, convex, centrally symmetric set $A$ of positive
volume with the property \[a\in A \Rightarrow |\Nm(a)| \le 1.\] Then every
$n$-dimensional lattice $\Lambda$ contains a nonzero point $x$ with \[|\Nm(x)|
\le \frac{2^n}{\vol(A)} \vol(\R^n/\Lambda).\]
\end{cor}
\begin{proof}
Apply the lemma with $E = tA$ where \[t^n = \frac{2^n}{\vol(A)}
\vol(\R^n/\Lambda).\]
\end{proof}

\begin{proof}[Proof of Theorem \ref{Minkowski}]
Let us get warmed up by proving a weaker result. Let $A$ be the set defined by
the inequalities \[|x_1| \le 1, \dots, |x_r| \le 1,
\underbrace{x_{r+1}^2+x_{r+2}^2 \le 1, \dots,x_{n-1}^2+x_{n}^2\le 1.}_{s \text{
inequalities}}\] Then $\vol(A) = 2^r\pi^s$ and we deduce from the previous
corollary that every $\Lambda$ contains a nonzero $x$ with \[|N(x)| \le \left(
\frac{4}{\pi} \right)^s \vol(\R^n/\Lambda).\]

Let us now proceed with the proof of the theorem as stated. Define $A$ by
\[|x_1| + \cdots + |x_r| + 2\left( \sqrt{x_{r+1}^2+x_{r+2}^2} + \cdots +
\sqrt{x_{n-1}^2+x_{n}^2} \right) \le n.\] We show that this set is convex.
Quite generally, let $f(x_1,\dots,x_n)$ be a continuous and nonnegative
function $\R^n \rightarrow \R_{\ge 0}$ and consider a set $E := \{x : f(x) \le
\alpha\} = f^{-1}([0,\alpha])$. If the condition \[\tag{$*$} x,y \in E
\Rightarrow (x+y)/2 \in E\] holds, then $E$ is convex. This follows from the
continuity of $f$ and the density in \R{} of the dyadic rationals (verify
this). Thus, to show that $A$ is convex, it suffices to show that $(*)$ holds
for $A$. This in turn follows from a combination of the triangle inequalities
for \R{} and $\R^2:$ \begin{align*}
|r+s| & \le |r| + |s| \\
\sqrt{(a+b)^2+(c+d)^2} & \le \sqrt{a^2+b^2}+\sqrt{c^2+d^2}.
\end{align*} (The last equality is just $\norm{x+y} \le \norm{x}+\norm{y}$ if
$x=(a,b),y=(c,d),\norm{(a,b)} = \sqrt{a^2+b^2}$.)

Recall the AM-GM ineqality for a finite sequence of nonnegative real numbers:
\[ (a_1 a_2 \cdots a_n)^{1/n} \le \frac{a_1 + a_2 + \cdots + a_n}{n}.\]

Now, the condition $a\in A \Rightarrow |\Nm(a)| \le 1$ follows from applying
AM-GM to the set of nonnegative real numbers
\[\left\{
|x_1|,\cdots, |x_r|,
\sqrt{x_{r+1}^2+x_{r+2}^2},
\sqrt{x_{r+1}^2+x_{r+2}^2},
\cdots
\sqrt{x_{n-1}^2+x_{n}^2},
\sqrt{x_{n-1}^2+x_{n}^2} \right\}
\] (note well the repetition). Namely, the GM is $|N(a)|^{1/n}$ and the AM is
at most 1.

Let us now show that \[\vol(A) = \frac{n^n}{n!} 2^r \left(
\frac{\pi}{2}\right)^s.\] Combining the above with the previous lemma and its
corollary will complete the proof of the theorem. First note that the case
$s=0, r=n$ of the computation of $\vol(A)$ is direct\footnote{In greater
detail, one starts with the $n$-dimensional cube defined by $\{x : \max |x_i|
\le n\}$ which clearly has volume $2^n n^n$. Now, $n!$ is the number of
permutations $\sigma$ of $\{1,\dots,n\}$. We cover the set $A$ with the $n!$
disjoint (up to sets of measure zero) and isometric sets \[A_\sigma :=
\{|x_{\sigma(1)}| \le \cdots \le |x_{\sigma(n)}| \};\] the permutation
corresponding to $x\in A$ is obtained by simply ordering the coordinates of
$x$ which shows that the $A_\sigma$ cover $A$.} and therefore we assume $s>0$
henceforth.

In general, let $V_{r,s}(t)$ be the volume of the subset of $\R^{r+2s}$ defined
by \[ |x_1| + \cdots + |x_r| + 2\left( \sqrt{x_{r+1}^2+x_{r+2}^2} + \cdots +
\sqrt{x_{n-1}^2+x_{n}^2} \right) \le t\] so that \[V_{r,s}(t) =
t^{r+2s}V_{r,s}(1).\] We claim that \[V_{r,s}(1) = \frac{1}{(r+2s)!} 2^r\left(
\frac{\pi}{2} \right)^s.\] For $r > 0$ we have:
\begin{align*}
  V_{r,s}(1) &= 2\int_0^1 V_{r-1,s}(1-x)dx\\
  &= 2V_{r-1,s}(1)\int_0^1 (1-x)^{r-1+2s}dx \\
  &= \frac{2}{r+2s}V_{r-1,s}(1).
\end{align*} Repeated application gives \[V_{r,s}(1) =
\frac{2^r}{(r+2s)(r+2s-1)\cdots (2s+1)} V_{0,s}(1).\] Now $s> 0$, and we
define\footnote{Of course, one should check that this is compatible with the
earlier computation\ldots} $V_{0,1}(1) := \pi/4$. We have \[V_{0,s}(1) =
\int_{\{x^2+y^2 \le 1/4\}} V_{0,s-1}\left( 1-2\sqrt{x^2+y^2}\right) dx dy.\]
Changing to polar coordinates we compute\footnote{There is a serious typo at
this point in \cite{M}.}
\begin{align*}
  V_{0,s}(1) &= \int_0^{2\pi} \int_0^{1/2} V_{0,s-1}(1-2\rho)
  \rho d\rho d\theta \\
  &= 2\pi \int_0^{1/2} (1-2\rho)^{2(s-1)}\rho V_{0,s-1}(1) d\rho
\end{align*} where we have used \(V_{r,s}(t) =
t^{r+2s}V_{r,s}(1).\) Thus, \begin{align*}
  V_{0,s}(1)/V_{0,s-1}(1) &= \frac{\pi}{2}\int_0^1 u^{2(s-1)}(1-u)du \\
  &= \frac{\pi}{2}\left( \frac{1}{2s-1} - \frac{1}{2s} \right)\\
  &= \frac{\pi}{2} \frac{1}{2s(2s-1)}
\end{align*}
and we obtain \[V_{0,s}(1) = \left( \frac{\pi}{2} \right)^{s}
\frac{1}{(2s)!}.\] The demonstrates our claim on the value of $\vol(A)$ and
completes the proof of the theorem. \end{proof}

\section{Dirichlet's Unit Theorem}

\begin{thm}
Write $U := \OK^\times$ and $r,2s$ for the number of real, resp.~complex
embeddings of $K$. Then $U_K$ is a finitely generated abelian group and we have
a direct product decomposition \[U = U_{\mathrm{tors}}U_{\mathrm{free}}.\] Here
$U_{\mathrm{tors}}$ coincides with the finite group of roots of unity of $K$
and $U_{\mathrm{free}}$ is a free abelian group of rank $r+s-1$ with \Z-basis
$\{u_i\}$ also known as a fundamental system of units.
\end{thm}
\begin{proof}
We have a sequence \[U \subset \OK- \{0\} \rightarrow \Lambda_{\OK}- \{0\}
\xrightarrow{\log} \R^{r+s} \] where $\log$ is defined in this context as
follows. For $(x_1,\dots,x_n)\in \Lambda_{\OK} - \{0\}$ \[\log(x_1,\dots,x_n)
:= (\log|x_1|,\dots,\log|x_r|,\log(x^2_{r+1}+x^2_{r+2}),\dots,
\log(x_{n-1}^2+x_n^2)).\] This is well defined since all of the arguments to
the classical logs are positive real numbers. We will also write log for the
composites $U\rightarrow \R^{r+s}$ and $\OK - \{0\} \rightarrow \R^{r+s}$. The
target, $\R^{r+s}$ shall be called the logarithmic space.

The following properties are immediate:

(1) $\log(\alpha \beta) = \log \alpha + \log \beta$ for $\alpha,\beta \in \R -
\{0\},$

(2) $\log U \subset H$ where $H\subset \R^{r+s}$ is the hyperplane defined by
$y_1+\cdots + y_{r+s} = 0$ --- this is because the norm of a unit is $\pm 1$.
The hyperplane $H$ is also called the ``trace zero hyperplane,''

(3) any bounded set in the logarithmic space has a finite preimage in $U$ ---
this is because preimage under log of a bounded set is still a bounded set and
$\Lambda_{\OK}$ is a lattice in the Euclidean space in question.

The first two properties show that $\log : U \rightarrow H$ is a group
homomorphism where $U$ is written multiplicatively and $H$ is written
additively being a real vector space. The third property shows that the kernel
in $U$ is equal to the set of roots of unity of $K$; namely, an element of the
kernel has finite order and is thus a root of unity and a root of unity lies on
the unit circle and is sent to zero by log. Moreover, we use the fact that
every finite subgroup of the circle is cyclic\footnote{Proof: if $x,y\in S^1$
are of finite order, then $x' = 2\pi/a, y' = 2\pi/b$ are lifts of $x,y$ in
\R{}, for some $a,b\in \Z_{> 0}$. Then the group generated by $x$ and $y$ is
the cyclic group on a generator whose lift is $2\pi/\mathrm{lcm}(a,b).$ This
is because we may obtain any common multiple of $a$ and $b$ in the
denominator by adding suitable multiples of $x'$ and $y'$.

Thus the set of roots of unity in a number field is a finite cyclic group.} to
deduce that the kernel is a cyclic group.

The third property also implies that the image $\log(U)$ is a discrete subgroup
of the logarithmic space. This is a general fact valid for subgroups of a real
vector space such that every bounded subset is finite\footnote{Proof: Choose a
bounded neighborhood of zero in the subgroup in question. It is finite hence
the topology on it induced from the ambient Euclidean space is discrete.
Hence 0 is both open and closed in this subgroup and thus it is discrete.

Warning: what is called a ``lattice'' on \cite[p.143]{M} is not the current
terminology. Lattice means discrete and cocompact subgroup, and what Marcus
calls ``lattice'' simply means discrete subgroup. Compare with the terminology
in \cite{MilneANT}.

The discussion on \cite[p.143]{M} says, in modern terminology, that an
\R-independent set in a Euclidean space gives rise, by taking integer linear
combinations, to a discrete subgroup, but on the other hand, a \Z-independent
set need not generate a discrete subgroup.}.

The discrete subgroup $\Lambda_U := \log(U)$ is contained in $H$ and thus is a
free abelian group of rank $d \le r+s-1.$

To see that $U$ is a direct product $U_{\mathrm{tors}}\cdot U_{\mathrm{free}}$,
we note that we have already seen that $U$ is generated by $U_{\mathrm{tors}}$
and representatives of the $d$ basis elements of $\Lambda_U$. Thus $U$ is a
finitely generated abelian group which therefore splits into the direct sum of
its torsion subgroup and a free part. Moreover, that free part is generated by
a splitting of the exact sequence \[0\rightarrow U_{\mathrm{tors}} \rightarrow
U \rightarrow \Lambda_U \rightarrow 0.\] Thus $U_{\mathrm{free}} := $ the
image of $\Lambda_U$ under the splitting.

It remains to show that $d=r+s-1$. We will do this by producing $r+s-1$ units
whose log vectors are \R-linearly independent. We need a lemma first.

\textsl{Lemma A. Fix $k$ with $1 \le k \le r+s$. For each nonzero
$\alpha\in \OK$ there exists a nonzero $\beta\in \OK$ with
\[|\Nm_{K/\Q}(\beta)| \le \left( \frac{2}{\pi} \right)^s
\sqrt{|\disc{\OK}|}\] such that if \begin{align*} \log(\alpha) &=
(a_1,\cdots,a_{r+s})\\ \log(\beta) &= (b_1,\cdots,b_{r+s})
\end{align*} then $b_i < a_i$ for all $i \neq k$.}

To see this, we use Blichtfeldt's Lemma (also called Minkowski's Lemma). Let
$E$ be the subset of $\R^n$ defined by the inequalities
\begin{gather*}
  |x_1| \le c_1, \dots , |x_r| \le c_r\\
  x_{r+1}^2 + x_{r+2}^2 \le c_{r+1} , \dots ,x_{n-1}^2 + x_n^2
  \le c_{r+s}
\end{gather*} where the $c_i$ are chosen to satisfy \(0 < c_i < e^{a_i}\) for
all $i\neq k$ and \[c_1 c_2 \cdots c_{r+s} = \left( \frac{2}{\pi} \right)^s
\sqrt{|\disc{\OK}|}. \] Then $\vol(E) = 2^r \pi^s c_1 \cdots c_{r+s} = 2^{n}
\vol(\R^n/\Lambda_{\OK}).$ Blichtfeldt then tells us that $E$ contains a
nonzero point of $\Lambda_{\OK}$ and one verifies that $\beta$ can be taken to
be the corresponding element of \OK. ///

Using Lemma A we can show that distinguished units exist.

\textsl{Lemma B. Fix $k$ with $1 \le k \le r+s$. Then there exists $u\in U$
such that if \[\log(u) = (y_1, \dots, y_{r+s})\] then $y_i < 0$ for all
$i\neq k$.}

Starting with a nonzero $\alpha_1\in \OK$ we apply Lemma A repeatedly to get a
sequence $\alpha_1,\alpha_2,\dots$ of nonzero elements of \OK{} with the
property that for each $i\neq k$ and for each $j\ge 1$ the $i$th coordinate of
$\log(\alpha_{j+1})$ is strictly less than that of $\log(\alpha_j)$ and
moreover that the numbers $|\Nm_{K/\Q}(\alpha_j)|$ are bounded. Then the
\norm{(\alpha_j)} are bounded. This implies (as in the proof of the finiteness
of class groups) that there are only finitely many distinct ideals
$(\alpha_j)$. Fixing any $j < h$ such that $(\alpha_j) = (\alpha_h)$, we have
$\alpha_h = \alpha_j u$ for some $u\in U.$ This is the sought element. ///

Lemma B shows that there are units $u_1,\dots,u_{r+s}$ such that all
coordinates of $\log(u_i)$ are negative except the $k$th. Since $\log(u_i)\in
H$ it follows that the $k$th coordinate is positive. We form the $(r+s)\times
(r+s)$ matrix having $\log(u_i)$ as its $i$th row. We claim that this matrix
has rank $r+s-1$ and hence that there are $r+s-1$ \R-linearly independent rows.
This will complete the proof of the unit theorem.

\textsl{Lemma C. Let $A = (a_{ij})$ be an $m\times m$ real matrix such that
$a_{ii} > 0$ for all $i$ and $a_{ij} < 0$ for all $i\neq j$ and such that,
moreover, each row-sum is 0 \[\sum_{j=1}^m a_{ij} = 0.\] Then $A$ has rank
$m-1$.}

Let us see that the first $m-1$ columns are linearly independent. Suppose not
so that $t_1 v_1 + \cdots + t_{m-1} v_{m-1} = 0$ where the $v_j$ are the column
vectors and the $t_j$ are real numbers, not all 0. Without loss we may assume
that some $t_k=1$ and that all other $t_j \le 1$ (divide by the coefficient
with maximum absolute value). Consider the $k$th row: \[0 = \sum_{j=1}^{m-1}
t_ja_{kj} \ge \sum_{j=1}^{m-1} a_{kj} > \sum_{j=1}^m a_{kj} = 0,\] a
contradiction. ///

This completes the proof of Dirichlet's unit theorem.
\end{proof}

\begin{egs}
(a) Let $K$ be imaginary quadratic. Then $r+s-1 = 0$ and $U =
K^{\times}_{\mathrm{tors}} = $ the finite group of roots of 1 in K.

(b) Let $K$ be real quadratic. Then $U = \{\pm u^k\}_{k\in \Z}$ is isomorphic
to $\Z/2 \oplus \Z.$ In this case, $u$ is called a fundamental unit in \OK.
\end{egs}

\bibliography{bib}
\end{document}

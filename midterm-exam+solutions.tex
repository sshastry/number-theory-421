\documentclass[10pt,a4paper,reqno]{amsart}

\usepackage{macros}
\usepackage{fullpage}

\begin{document}
\bibliographystyle{alpha}

\noindent \textit{Algebraic Number Theory, Math 421}

\noindent \textit{Instructor: Sreekar M. Shastry}

\noindent \textit{Solutions to the Mid Semester Examination}

\bigskip

\begin{itemize}
  \item Choose any 5 problems. Each problem is worth 5 points.
  \item The maximum score is 25 points.
  \item It is not permitted to submit partial solutions to more than 5
    problems; you must choose the 5 that you want to be graded.
  \item Clearly state the results which you invoke.
\end{itemize}

\bigskip

1. Let $K := \Q[\sqrt{m}]$ where $m$ is squarefree and suppose that $p$ is an
odd prime. Write $\left( \frac{m}{p} \right)$ for the Legendre symbol. Show
that
\[ \left( \frac{m}{p} \right) =
\begin{cases}
    1 & \text{if } p.\OK = \text{ product of distinct primes of } \OK\\
    0 & \text{if } p.\OK = \text{ the square of a prime of } \OK\\
    -1 & \text{if } p.\OK = \text{ a prime in } \OK
\end{cases} \]

\textit{Solution.} We will use Theorem 27 on page 79 of Marcus which relates
the factorization of $p.\OK$ to the reduction mod $p$ of the polynomial $X^2 -
m\in \Z[X]$. This theorem is sometimes called Kummer's theorem.

First of all, note that $\OK = \Z[\alpha]$ with $\alpha = \sqrt{m}$ or
$(1+\sqrt{m})/2$ according as $m\equiv -1,2\imod{4}$ or $m \equiv 1\imod{4}$
(see page 30 of Marcus).

We are considering in all cases the ring $\Z[\sqrt{m}]$ and we must check that
$p$ is prime to $\#\OK/\Z[\sqrt{m}]$ (quotient of additive abelian groups) when
$m\equiv 1 \imod{4}$. We have that $\Z[\sqrt{m}] = \Z[1+\sqrt{m}]$ which makes
it clear that the quotient group in question is of order 2; $p$ is an odd prime
and so we may proceed.

Now, $\left( \frac{m}{p} \right) = 1 \Longleftrightarrow X^2-m \imod{p}$ is
reducible $\Longleftrightarrow p.\OK$ is a product of distinct primes by the
Theorem; $e=f=1, r=2.$

Next, $\left( \frac{m}{p} \right) = -1 \Longleftrightarrow X^2-m \imod{p}$ is
irreducible $\Longleftrightarrow f(Q/p) = 2, e = r = 1$ so that $p.\OK = Q$ is
a prime. Again we have used the theorem.

Finally, $\left( \frac{m}{p} \right) = 0 \Longleftrightarrow p|m
\Longleftrightarrow X^2-m \equiv X^2 \imod{p}$ is the square of the prime $(X)
\subset \mathbb{F}_p[X]$.

\bigskip

2. The objective of this problem is to show that every prime $p\equiv 1
\imod{4}$ is a sum of two squares.

(a) Use the fact that $(\Z/p)^\times$ is cyclic to show that if $p\equiv 1
\imod{4}$ then $n^2 \equiv -1 \imod{p}$ for some $n\in \Z.$

(b) Show that $p$ cannot be irreducible in $\Z[i]$. (Hint: use (a).)

(c) Prove that $p$ is a sum of two squares. (Hint: use (b).)

\bigskip

\textit{Solution.} (a) The group $(\Z/p)^\times$ is cyclic of order $p-1$ and
$p-1$ is divisible by 4, and thus we consider the homomorphism $x\mapsto
x^\frac{p-1}{4}$ from $(\Z/p)^\times$ to itself. Let $y = x_0^{\frac{p-1}{4}}$
be in the image of this homomorphism. Then $y^2 \equiv -1 \imod{4}$ since
$(y^{2})^{2} = x_0^{p-1} = 1$ and on the other hand $-1 \in (\Z/p)^\times$ is
the unique element $g$ such that $g^2 = 1$. Here we have used the cyclicity.
Now choose $n\in\Z$ to be any lift of $y$.

(b) This follows from $p | n^2 + 1 = (n+i)(n-i)$ in $\Z[i]$. Here we must use
the fact that $\Z[i]$ is a UFD.

(c) Recall that $\Z[i]$ is a Euclidean domain hence PID hence UFD. We have $p =
(a+bi)(c+di)$ with neither factor a unit, by (b) --- if no such expression was
possible, i.e. if in every such factorization one of the factors was a unit,
then $p$ would still be a prime in $\Z[i]$ --- that shows that $p$ has at least
two non unit factors; that there are at most two follows since the norm of each
nonunit factor is a nontrivial divisor of $\Nm(p) = p^2$. On the other hand,
the ideal $p.\Z[i]$ is principle and generated by a nonzero element
$\alpha+\beta i$ of minimal norm. The norm of this element is a proper and
nontrivial divisor of $\Nm(p) = p^2$, hence $p$, and we have $\Nm(\alpha+\beta
i) = p = \alpha^2 + \beta^2$.

\bigskip

3. Let $\omega := e^{2\pi i/p}$ with $p$ an odd prime. Show that $\Q[\omega]$
contains $\sqrt{p}$ if $p\equiv 1 \imod{4}$ and contains $\sqrt{-p}$ if
$p\equiv -1 \imod{4}$.

\bigskip

\textit{Solution.} We shall make use of the fact that the discriminant of the
field extension $\Q[\omega]/\Q$ is $\Delta := (-1)^{\frac{p-1}{2}}p^{p-2}$. We
may use the Vandermonde determinant to compute the discriminant as well:
\[\Delta = \prod_{\substack{i< j\\
\sigma_i \in \Gal{\Q[\omega]/\Q}}}
(\sigma_i(\omega)-\sigma_j(\omega))^{2}\] to conclude that $\Delta = x^2 \in
K$, i.e.~that $\Delta$ is a square in $K$ and hence that
$\sqrt{(-1)^{\frac{p-1}{2}}p^{p-2}} \in K$. We multiply the last expression by
$1/p^{\frac{p-3}{2}}$ (which is in $K$ because $p$ is odd) to see that
$\sqrt{(-1)^{\frac{p-1}{2}}p} \in K$. This completes the proof.

\bigskip

4. (a) Show that if $m$ is squarefree, $m <0$, and $m\neq -1, -3$, then $\pm 1$
are the only units in the ring of integers of $\Q[\sqrt{m}]$.

(b) What if $m = -1$ or $-3?$

\bigskip

\textit{Solution.} (a) Two cases.

Case 1. $m\equiv -1,2\imod{4}.$ In this case the integral basis of the ring of
integers is given by $\{1,\sqrt{m}\}$ and the norm is $\Nm(a+b\sqrt{m}) =
a^2-b^2 m = a^2+|m|b^2$ (since $m< 0$). We set this equal to 1 and solve (no
need to check the $-1$ case as the norm is automatically positive). We get
$a=\pm 1, b = 0$.

Case 2. $m\equiv 1 \imod{4}.$ The basis is $\{1,(1+\sqrt{m})/2\}$ and a typical
element is $\frac{a+b\sqrt{m}}{2}$ with $a\equiv b \imod{2}$ and we similarly
reduce to solving the equation \[a^2+|m|b^2 = 4.\] Solving gives $a = \pm 2,
b=0$ as required.

(b) $m=-1.$ This is the fundamental example of the Gaussian integers $\Z[i]$.
In this case, we quickly calculate with norms to see that the units are $\{\pm
1, \pm i\}.$

$m=-3.$ The equation to solve is $a^2+3b^2 = 4$ and the units are $\{\pm 1,(\pm
1\pm \sqrt{-3})/2\}.$ (The factor of 1/2 comes from the presentation of the
ring of integers when $m\equiv 1 \imod{4}).$

\bigskip

5. Let $\alpha$ be an algebraic integer and let $f$ be any monic polynomial in
$\Z[x]$ such that $f(\alpha)=0$. Show that \( \disc{\alpha} \text{ divides }
\Nm_{\Q[\alpha]/\Q}(f'(\alpha)).\)

\bigskip

\textit{Solution.} We shall invoke Theorem 8 on page 26 of Marcus which tells
us that the discriminant of $\Q[\alpha]/\Q$ is $\pm
\Nm_{\Q[\alpha]/\Q}(F'(\alpha))$ where $F$ is the minimal polynomial of
$\alpha$.

In our case, $f$ is not necessarily the minimal polynomial, and all we know is
that $f(X) = F(X)g(X)$ where $F$ is the minimal polynomial of $\alpha$. Taking
derivatives gives us $f'(X) = F'(X) g(X) + g'(X) F(X)$ and evaluating at
$\alpha$ gives us \[f'(\alpha) = F'(\alpha) g(\alpha).\]

Taking norms we have $\Nm(f'(\alpha)) = \Nm(F'(\alpha)) \Nm(g(\alpha)) =
\disc{\alpha}.c$ where $c \in \Z$. The last assertion holds because
$g(\alpha)$, being an algebraic integer, has norm in $\Z$.

\bigskip

6. Let $K$ be a number field and $\mathfrak{a} \subset \OK$ be a nonzero ideal.
Show that $\# (\OK/\mathfrak{a})$ divides $\Nm_{K/\Q}(\alpha)$ for all
$\alpha\in \mathfrak{a}$, and equality holds iff $\mathfrak{a} = (\alpha)$.

\bigskip

\textit{Solution.} We shall use Theorem 22 (c) on page 66 of Marcus which tells
us that $\#\OK/(\alpha) = |\Nm_{K/\Q}(\alpha)|.$

The tower of abelian group $(\alpha) \subset \mathfrak{a} \subset \OK$ gives us
\[ [\OK : (\alpha)] = [\OK: \mathfrak{a}] [\mathfrak{a} : (\alpha)]\] and since
the left hand side equals $\#\OK/(\alpha)$, the problem is solved.

\bigskip

7. Let $L/K$ be a Galois extension with group $G$. Fix a prime $P$ of $K$ and a
prime $Q$ of $L$ which lies above it.

(a) Define the inertia and decomposition groups associated to $Q/P$.  What is
the relation between these groups and the Galois group of the residual
extension associated to $Q/P$? Give a proof of your assertion.

(b) Let $P$ be a prime of $K$ and $Q$ be a prime of $L$ above it.  Define
$\Fr{Q/P}$. If $G$ is abelian, what more can you say?  Give a proof of your
assertion.

(c) Show that for all $\sigma\in G,$ we have \[\Fr{\sigma Q/P} = \sigma
\Fr{Q/P} \sigma^{-1}.\]

\bigskip

\textit{Solution.} The solutions to (a) and (b) may be found in the
book.

For (c) we compute as follows.  Let $x\in \OL$, and let $k(P) := \OK/P$.

(1) $\Fr{Q/P}(x) \equiv x^{\# k(P)} \imod{Q}.$

(2) $\Fr{\sigma Q/P}(x) \equiv x^{\# k(P)} \imod{\sigma Q}.$

(3) $\Fr{Q/P}(\sigma^{-1}(x)) \equiv (\sigma^{-1}(x))^{\# k(P)}
\imod{Q}. $

(4) $\sigma(\Fr{Q/P}(\sigma^{-1}(x))) \equiv x^{\# k(P)} \equiv
\Fr{\sigma Q/P}(x) \imod{\sigma Q}.$

(5) $\Fr{\sigma Q/P} = \sigma \Fr{Q/P} \sigma^{-1}.$

(1), (2), (3) are just the definitions. We apply $\sigma$ to (3) to
get (4) and (4) is the same as (5) which is what we were looking for.

\bigskip

8. Let $L/K$ be a Galois extension of number fields with group $G$ and let $P$
be a prime of $K$. By ``intermediate field'' we mean ``intermediate field
different from $K$ and $L$.''

(a) Show that if $P$ is inert in $L$ then $G$ is cyclic. In other words, show
that no prime remains inert in a non-cyclic Galois extension of number fields.

(b) Suppose that $P$ is totally ramified in every intermediate field, but not
totally ramified in $L$. Show that no intermediate field can exist. What can
you say about the structure of $G$ in this case?

\bigskip

\textit{Solution.} We make use of the following theorem, written in standard
notation, especially $e := e(Q/P), f:= f(Q/P), I:= I(Q/P), D := D(Q/P)$. We
have
\[
\xymatrix{
L \ar@{-}[d]_{e = [L:L^I]} & Q \ar@{-}[d]^{e(Q/Q^I) = e,\, f(Q/Q^I)=1} \\
L^I \ar@{-}[d]_{f=[L^I:L^D]} & Q^I \ar@{-}[d]^{e(Q^I/Q^D)=1,\,f(Q^I/Q^D)=f} \\
L^D \ar@{-}[d]_{r=[L^D:K]} & Q^D \ar@{-}[d]^{e(Q^D/P)=f(Q^D/P)=1} \\
K & P.
}
\] This is Theorem 28 page 100 of Marcus.

(a) The fact that $P$ is inert gives us $e=r=1, f=n$. Thus $L^D = L$ and $L^I =
L$.  Thus $I = \{1\}$ and $G = D.$ But $D/I$ is always cyclic. Thus $G$ is
cyclic.

(b) Since $P$ totally ramifies in any intermediate field, $L^I$ cannot be an
intermediate field (since $L^I$ is the maximal intermediate field in which $P$
does not ramify). Thus $L^I = L$ or $K$. In fact $L^I = K$ because otherwise
$P$ would be unramified in $L$ while simultaneously being totally ramified in
every intermediate field.  Since the ramification index is multiplicative in
towers, this cannot occur.

Thus $e=n$ and $P$ is totally ramified in $L$, contradicting our hypothesis.

Therefore the only way that $P$ could be totally ramified in every intermediate
field and not totally ramified in $L$ is if there are no intermediate fields.

This means that $G$ has no proper subgroups at all and is hence cyclic of prime
order.

\end{document}
